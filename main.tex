% !TeX document-id = {20a758e9-3325-489d-88bb-4096b1ed6a15}
% !TeX spellcheck = en-US
% !TeX encoding = utf8
% !TeX program = pdflatex
% !BIB program = biber
% -*- coding:utf-8 mod:LaTeX -*-


% vv  scroll down to line 200 for content  vv


\let\ifdeutsch\iffalse

\let\ifenglish\iftrue


\input{pre-documentclass}
\documentclass[
  %
  %ngerman, %%% Add if you write in German.
  %
  % fontsize=11pt is the standard
  a4paper,  % Standard format - only KOMAScript uses paper=a4 - https://tex.stackexchange.com/a/61044/9075
  twoside,  % we are optimizing for both screen and two-side printing. So the page numbers will jump, but the content is configured to stay in the middle (by using the geometry package)
  bibliography=totoc,
  %               idxtotoc,   %Index ins Inhaltsverzeichnis
  %               liststotoc, %List of X ins Inhaltsverzeichnis, mit liststotocnumbered werden die Abbildungsverzeichnisse nummeriert
  headsepline,
  cleardoublepage=empty,
  parskip=half,
  %               draft    % um zu sehen, wo noch nachgebessert werden muss - wichtig, da Bindungskorrektur mit drin
  draft=false
]{scrbook}
\input{config}


\usepackage[
  title={Trustworthiness of Synthetic Media in the Context of a Newscast}, % Do not forget to capitalize your title correctly, you may use the following page to help you: https://capitalizemytitle.com/
  author={Danilo Pejakovic},
  %orcid=0000-0000-0000-0000, % get your own ORCID via https://orcid.org/
  email={danilo.pejakovic@campus.lmu.de},
  type={Masterthesis},
  institute={Institute for Informatics}, % or other institute names - or just a plain string using {Demo\\Demo...}
  course={Mediainformatics},
  examiner={Prof.\ Dr.\ Sylvia Rothe, Christoph Weber},
  supervisor={Prof.\ Dr.\ Sylvia Rothe},
  startdate={October 4, 2023},
  enddate={April 14, 2024},
  % Falls keine Lizenz gewünscht wird bitte auf "none" setzen
  % Die Lizenz erlaubt es zu nichtkommerziellen Zwecken die Arbeit zu
  % vervielfältigen und Kopien zu machen. Dabei muss aber immer der Autor
  % angegeben werden. Eine kommerzielle Verwertung ist für den Autor
  % weiter möglich.
  copyright=ccbysa, % ccbysa, ccbynosa, cc0, none
  language=english
]{lmu-thesis-cover}

% Hier stehen alle Abkürzungen
% \newacronym{label}{abbrv}{long}
\newacronym{er}{ER}{error rate}
\newacronym{fr}{FR}{Fehlerrate}
\newacronym{w2l}{W2L}{Wav2Lip}
\newacronym{tts}{TTS}{text to speech}
\newacronym{t2i}{t2i}{text to image}
\newacronym{rvc}{RVC}{Retrieval Based Voice Conversion}
\newacronym{br}{BR}{bayerischer Rundfunk}
\newacronym{sd}{SD}{Stable Diffusion}
\newacronym{genai}{gen. A.I.}{Generative Artificial Intelligence}
\newacronym{oss}{OSS}{Open Source Software}
\newacronym{dfl}{DFL}{DeepFaceLab}

\geometry{
  left=2.5cm,
  right=3.5cm,
  top=2cm,
  bottom=2cm
}

\makeindex

\begin{document}

\frontmatter
\pagenumbering{roman} % Seitennummerierung mit römischen Ziffern für den Vorspann
\setcounter{tocdepth}{3} % bis zur dritten Gliederungsebene Anzeigen



%tex4ht-Konvertierung verschönern
\iftex4ht
  % tell tex4ht to create picures also for formulas starting with '$'
  % WARNING: a tex4ht run now takes forever!
  \Configure{$}{\PicMath}{\EndPicMath}{}
  %$ % <- syntax highlighting fix for emacs
  \Css{body {text-align:justify;}}

  %conversion of .pdf to .png
  \Configure{graphics*}
  {pdf}
  {\Needs{"convert \csname Gin@base\endcsname.pdf
      \csname Gin@base\endcsname.png"}%
    \Picture[pict]{\csname Gin@base\endcsname.png}%
  }
\fi

%\VerbatimFootnotes %verbatim text in Fußnoten erlauben. Geht normalerweise nicht.

\input{commands}
%\pagenumbering{arabic}
\Coverpage
\Copyright
%Eigener Seitenstil fuer die Kurzfassung und das Inhaltsverzeichnis
\deftriplepagestyle{preamble}{}{}{}{}{}{\pagemark}
%Doku zu deftriplepagestyle: scrguide.pdf
\pagestyle{preamble}
\renewcommand*{\chapterpagestyle}{preamble}



%Kurzfassung / abstract
%auch im Stil vom Inhaltsverzeichnis

\section*{Abstract}

\todo{Short summary of the thesis. Here, the following questions should be answered:}
\todo{What is the specific problem addressed?}
\todo{What have you done?}
\todo{What did you find out?}
\todo{What are the implications on a larger scale?}
\todo{Should be around 0.5 pages. Not longer than 1 page.}

\cleardoublepage


% BEGIN: Verzeichnisse

\iftex4ht
\else
  \microtypesetup{protrusion=false}
\fi

%%%
% Literaturverzeichnis ins TOC mit aufnehmen, aber nur wenn nichts anderes mehr hilft!
% \addcontentsline{toc}{chapter}{Literaturverzeichnis}
%
% oder zB
%\addcontentsline{toc}{section}{Abkürzungsverzeichnis}
%
%%%

%Produce table of contents
%
%In case you have trouble with headings reaching into the page numbers, enable the following three lines.
%Hint by http://golatex.de/inhaltsverzeichnis-schreibt-ueber-rand-t3106.html
%
%\makeatletter
%\renewcommand{\@pnumwidth}{2em}
%\makeatother
%
\tableofcontents

% Bei einem ungünstigen Seitenumbruch im Inhaltsverzeichnis, kann dieser mit
% \addtocontents{toc}{\protect\newpage}
% an der passenden Stelle im Fließtext erzwungen werden.

\listoffigures
\listoftables


% Control List of Listings
\let\iflistings\iffalse
%Wird nur bei Verwendung von der lstlisting-Umgebung mit dem "caption"-Parameter benoetigt
%\lstlistoflistings
%ansonsten:
\iflistings
  \ifdeutsch
    \listof{Listing}{Verzeichnis der Listings}
  \else
    \listof{Listing}{List of Listings}
  \fi
\fi

% Control List of Algorithms
\let\ifalgorithms\iffalse
\ifalgorithms
  %mittels \newfloat wurde die Algorithmus-Gleitumgebung definiert.
  %Mit folgendem Befehl werden alle floats dieses Typs ausgegeben
  \ifdeutsch
    \listof{Algorithmus}{Verzeichnis der Algorithmen}
  \else
    \listof{Algorithmus}{List of Algorithms}
  \fi
  %\listofalgorithms %Ist nur für Algorithmen, die mittels \begin{algorithm} umschlossen werden, nötig
\fi

% Control Glossary
\let\ifglossary\iftrue
\ifglossary
  \printnoidxglossaries
\fi

\iftex4ht
\else
  %Optischen Randausgleich und Grauwertkorrektur wieder aktivieren
  \microtypesetup{protrusion=true}
\fi

% END: Verzeichnisse


% Headline and footline
\renewcommand*{\chapterpagestyle}{scrplain}
\pagestyle{scrheadings}
\pagestyle{scrheadings}
\ihead[]{}
\chead[]{}
\ohead[]{\headmark}
\cfoot[]{}
\ofoot[\usekomafont{pagenumber}\thepage]{\usekomafont{pagenumber}\thepage}
\ifoot[]{}


%% vv  scroll down for content  vv %%

\mainmatter

%%%%%%%%%%%%%%%%%%%%%%%%%%%%%%%%%%%%%%%%%%%%%%%%%%%%%%%%%%%%%%%%%%%%%%%%%%%%%%
%
% Main content starts here
%
%%%%%%%%%%%%%%%%%%%%%%%%%%%%%%%%%%%%%%%%%%%%%%%%%%%%%%%%%%%%%%%%%%%%%%%%%%%%%%


\chapter{Introduction}
\label{sec:introduction}


%\todo{P1.1. What is the large scope of the problem?}
\begin{quotation}
"Sophiscitcated AI systems are increasingly everywhere. [...] However, 2023 will likely prove to be a particularly critical moment in the history of AI" (\citet{arguedasAutomatingDemocracyGenerative2023}).
\end{quotation}
This paper and corresponding study are being conducted in the very year of 2023. As the previously quoted authors state, we might be experiencing a tippping point in AI development, as more and more tools become available to a broader user base. These developments are tightly linked to the rise of OpenAi's ChatGPT and other, widely adopted technologies like \gls{sd} based \gls{t2i} generators. These developments are currently summarized as \gls{genai}: "Generative AI is an umbrella term used for AI systems that can generate new forms of data, often by applying machine learning to large quantities of training data" (\citet{arguedasAutomatingDemocracyGenerative2023}). One could extend this definition with the following: Besides just generating new forms of data, \gls{genai} can be used to augment, reduce, manipulate and mix real data with the generated data in such a form, that it is impossible to distinguish between real and syntheticly generated (fake), or anything in between on that spectrum.

\todo{P1.2. What is the specific problem?}

In the context of media production and distribution the developments of \gls{genai} open up an important discussion about trust and credibility. For the sake of completeness, these discussions are not entirely new: So called DeepFakes (blend word of Deep learning and Fake News) have been around quite some time. First research papers like the Face2Face approach by \citet{thiesFace2FaceRealtimeFace2020} date back to the year 2016. 


Surely AI generated Content now is very much in the focus of public attention.



Case, Sievers fake. 
Telegram. 
Artificial voices on Youtube.
Radio Station




% Second Paragraph
% CORE MESSAGE OF THIS PARAGRAPH:
\todo{P2.1. The second paragraph should be about what have others been doing}
\todo{P2.2. Why is the problem important? Why was this work carried out?}

% Third Paragraph
% CORE MESSAGE OF THIS PARAGRAPH:
\todo{P3.1. What have you done?}
\todo{P3.2. What is new about your work?}

% Fourth paragraph
% CORE MESSAGE OF THIS PARAGRAPH:
\todo{P4.1. What did you find out? What are the concrete results?}
\todo{P4.2. What are the implications? What does this mean for the bigger picture?}

LaTeX hints are provided in \autoref{chap:latexhints}.

\chapter{Related Work}
Trust in news is explored. 
But there is not too much about trust and credibility in news with ai footage.
Law findings. 
Verletzung Persönlichkeitsrecht. 
Trainingsmaterial nutzung? 

Not too much implementation Details about the used technology

\todo{reception of media}
\todo{synthetic media reception}
\todo{uncanny valley}

\chapter{Implemented Technologies}
Docker based deployment, CI/CD. Remote work on HFF Workstation

\section{Voice Cloning Toolkit}

\subsection{Preparation and Pre-Processing}
\subsection{Text To Speech}
\subsection{Retrieval Based Voice Conversion}

\section{Lip remapping}
Wav2Lip

\section{stable Diffusion + Video}

\section{Game-Engine: Unreal Engine}
Einzelpraktikum stuff

\chapter{Study Design}
Spectrum of syntheticnes.
Randomization.
AB Groups. 
Sequence bias.
Basic Questions: Media usage, media trust, ai usage and trust

\section{Procedure and Participants}
Online survey, spread via various channels.


\chapter{Results}

\chapter{Discussion}

\chapter{Conclusion}

Future implementation und Haushandlungsprozess. Was wichtig ist:) 

To close the circle with the beginning. This work might be too early. The disruptive process has just begun, the developments can happen rather quickly. It therefore might remain interesting to closely monitor how the credibility and trust toward media, both synthetic and real, unfolds in the next years. 



\label{sec:conclusion}

\todo{Outlook}

\printbibliography

All links were last followed on \today{}.

\appendix
\input{latexhints/latexhints-english}

\pagestyle{empty}
\renewcommand*{\chapterpagestyle}{empty}
\Affirmation
\end{document}
