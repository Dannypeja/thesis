% !TeX document-id = {20a758e9-3325-489d-88bb-4096b1ed6a15}
% !TeX spellcheck = en-US
% !TeX encoding = utf8
% !TeX program = pdflatex
% !BIB program = biber
% -*- coding:utf-8 mod:LaTeX -*-


% vv  scroll down to line 200 for content  vv


\let\ifdeutsch\iffalse

\let\ifenglish\iftrue


\input{pre-documentclass}
\documentclass[
  %
  %ngerman, %%% Add if you write in German.
  %
  % fontsize=11pt is the standard
  a4paper,  % Standard format - only KOMAScript uses paper=a4 - https://tex.stackexchange.com/a/61044/9075
  twoside,  % we are optimizing for both screen and two-side printing. So the page numbers will jump, but the content is configured to stay in the middle (by using the geometry package)
  bibliography=totoc,
  %               idxtotoc,   %Index ins Inhaltsverzeichnis
  %               liststotoc, %List of X ins Inhaltsverzeichnis, mit liststotocnumbered werden die Abbildungsverzeichnisse nummeriert
  headsepline,
  cleardoublepage=empty,
  parskip=half,
  %               draft    % um zu sehen, wo noch nachgebessert werden muss - wichtig, da Bindungskorrektur mit drin
  draft=false
]{scrbook}
\input{config}


\usepackage[
  title={Trustworthiness of Synthetic Media in the Context of a Newscast}, % Do not forget to capitalize your title correctly, you may use the following page to help you: https://capitalizemytitle.com/
  author={Danilo Pejakovic},
  %orcid=0000-0000-0000-0000, % get your own ORCID via https://orcid.org/
  email={danilo.pejakovic@campus.lmu.de},
  type={Masterthesis},
  institute={Institute for Informatics}, % or other institute names - or just a plain string using {Demo\\Demo...}
  course={Mediainformatics},
  examiner={Prof.\ Dr.\ Sylvia Rothe, Christoph Weber},
  supervisor={Prof.\ Dr.\ Sylvia Rothe},
  startdate={October 4, 2023},
  enddate={April 14, 2024},
  % Falls keine Lizenz gewünscht wird bitte auf "none" setzen
  % Die Lizenz erlaubt es zu nichtkommerziellen Zwecken die Arbeit zu
  % vervielfältigen und Kopien zu machen. Dabei muss aber immer der Autor
  % angegeben werden. Eine kommerzielle Verwertung ist für den Autor
  % weiter möglich.
  copyright=ccbysa, % ccbysa, ccbynosa, cc0, none
  language=english
]{lmu-thesis-cover}

% Hier stehen alle Abkürzungen
% \newacronym{label}{abbrv}{long}
\newacronym{er}{ER}{error rate}
\newacronym{fr}{FR}{Fehlerrate}
\newacronym{w2l}{W2L}{Wav2Lip}
\newacronym{tts}{TTS}{text to speech}
\newacronym{t2i}{t2i}{text to image}
\newacronym{rvc}{RVC}{Retrieval Based Voice Conversion}
\newacronym{br}{BR}{bayerischer Rundfunk}
\newacronym{sd}{SD}{Stable Diffusion}
\newacronym{genai}{gen. A.I.}{Generative Artificial Intelligence}
\newacronym{oss}{OSS}{Open Source Software}
\newacronym{dfl}{DFL}{DeepFaceLab}

\geometry{
  left=2.5cm,
  right=3.5cm,
  top=2cm,
  bottom=2cm
}

\makeindex

\begin{document}

\frontmatter
\pagenumbering{roman} % Seitennummerierung mit römischen Ziffern für den Vorspann
\setcounter{tocdepth}{3} % bis zur dritten Gliederungsebene Anzeigen



%tex4ht-Konvertierung verschönern
\iftex4ht
  % tell tex4ht to create picures also for formulas starting with '$'
  % WARNING: a tex4ht run now takes forever!
  \Configure{$}{\PicMath}{\EndPicMath}{}
  %$ % <- syntax highlighting fix for emacs
  \Css{body {text-align:justify;}}

  %conversion of .pdf to .png
  \Configure{graphics*}
  {pdf}
  {\Needs{"convert \csname Gin@base\endcsname.pdf
      \csname Gin@base\endcsname.png"}%
    \Picture[pict]{\csname Gin@base\endcsname.png}%
  }
\fi

%\VerbatimFootnotes %verbatim text in Fußnoten erlauben. Geht normalerweise nicht.

\input{commands}
%\pagenumbering{arabic}
\Coverpage
\Copyright
%Eigener Seitenstil fuer die Kurzfassung und das Inhaltsverzeichnis
\deftriplepagestyle{preamble}{}{}{}{}{}{\pagemark}
%Doku zu deftriplepagestyle: scrguide.pdf
\pagestyle{preamble}
\renewcommand*{\chapterpagestyle}{preamble}



%Kurzfassung / abstract
%auch im Stil vom Inhaltsverzeichnis

\section*{Abstract}

\todo{Short summary of the thesis. Here, the following questions should be answered:}
\todo{What is the specific problem addressed?}
\todo{What have you done?}
\todo{What did you find out?}
\todo{What are the implications on a larger scale?}
\todo{Should be around 0.5 pages. Not longer than 1 page.}

\cleardoublepage


% BEGIN: Verzeichnisse

\iftex4ht
\else
  \microtypesetup{protrusion=false}
\fi

%%%
% Literaturverzeichnis ins TOC mit aufnehmen, aber nur wenn nichts anderes mehr hilft!
% \addcontentsline{toc}{chapter}{Literaturverzeichnis}
%
% oder zB
%\addcontentsline{toc}{section}{Abkürzungsverzeichnis}
%
%%%

%Produce table of contents
%
%In case you have trouble with headings reaching into the page numbers, enable the following three lines.
%Hint by http://golatex.de/inhaltsverzeichnis-schreibt-ueber-rand-t3106.html
%
%\makeatletter
%\renewcommand{\@pnumwidth}{2em}
%\makeatother
%
\tableofcontents

% Bei einem ungünstigen Seitenumbruch im Inhaltsverzeichnis, kann dieser mit
% \addtocontents{toc}{\protect\newpage}
% an der passenden Stelle im Fließtext erzwungen werden.

\listoffigures
\listoftables


% Control List of Listings
\let\iflistings\iffalse
%Wird nur bei Verwendung von der lstlisting-Umgebung mit dem "caption"-Parameter benoetigt
%\lstlistoflistings
%ansonsten:
\iflistings
  \ifdeutsch
    \listof{Listing}{Verzeichnis der Listings}
  \else
    \listof{Listing}{List of Listings}
  \fi
\fi

% Control List of Algorithms
\let\ifalgorithms\iffalse
\ifalgorithms
  %mittels \newfloat wurde die Algorithmus-Gleitumgebung definiert.
  %Mit folgendem Befehl werden alle floats dieses Typs ausgegeben
  \ifdeutsch
    \listof{Algorithmus}{Verzeichnis der Algorithmen}
  \else
    \listof{Algorithmus}{List of Algorithms}
  \fi
  %\listofalgorithms %Ist nur für Algorithmen, die mittels \begin{algorithm} umschlossen werden, nötig
\fi

% Control Glossary
\let\ifglossary\iftrue
\ifglossary
  \printnoidxglossaries
\fi

\iftex4ht
\else
  %Optischen Randausgleich und Grauwertkorrektur wieder aktivieren
  \microtypesetup{protrusion=true}
\fi

% END: Verzeichnisse


% Headline and footline
\renewcommand*{\chapterpagestyle}{scrplain}
\pagestyle{scrheadings}
\pagestyle{scrheadings}
\ihead[]{}
\chead[]{}
\ohead[]{\headmark}
\cfoot[]{}
\ofoot[\usekomafont{pagenumber}\thepage]{\usekomafont{pagenumber}\thepage}
\ifoot[]{}


%% vv  scroll down for content  vv %%

\mainmatter

%%%%%%%%%%%%%%%%%%%%%%%%%%%%%%%%%%%%%%%%%%%%%%%%%%%%%%%%%%%%%%%%%%%%%%%%%%%%%%
%
% Main content starts here
%
%%%%%%%%%%%%%%%%%%%%%%%%%%%%%%%%%%%%%%%%%%%%%%%%%%%%%%%%%%%%%%%%%%%%%%%%%%%%%%


\chapter{Introduction 3 Seiten}
\label{sec:introduction}


%\todo{P1.1. What is the large scope of the problem?}
\begin{quotation}
"Sophiscitcated AI systems are increasingly everywhere. [...] However, 2023 will likely prove to be a particularly critical moment in the history of AI" \citet{arguedasAutomatingDemocracyGenerative2023}.
\end{quotation}
This paper and corresponding study are being conducted in the very year of 2023. As the previously quoted authors state, we might be experiencing a tippping point in AI development, as more and more tools become available to a broader user base. These developments are tightly linked to the rise of OpenAi's ChatGPT and other, widely adopted technologies like \gls{sd} based \gls{t2i} generators. These developments are currently summarized as \gls{genai}: "Generative AI is an umbrella term used for AI systems that can generate new forms of data, often by applying machine learning to large quantities of training data" \citet{arguedasAutomatingDemocracyGenerative2023}. One could extend this definition with the following: Besides just generating new forms of data, \gls{genai} can be used to augment, reduce, manipulate and mix real data with the generated data in such a form, that it is impossible to distinguish between real and syntheticly generated (fake), or anything in between on that spectrum.
%\todo{P1.2. What is the specific problem?}
In the context of media production and and media distribution the developments of \gls{genai} open up an important discussion about trust and credibility. Legitimate media, has always been using synthetic content for demonstration or illustration purposes. One can just think of animated explainatory videos, or other infographics. The difference is, that in those illustrations it was quite clear, that these images were "not real". Therefore the question for legitimate media outlets remains, of how synthetic content will be recieved among the audience. Additionally, just the term "A.I." sparks criticism. These effects on audience, their mitigation, and at the same time, education of the broader public about technologic advancements are very interesting topics for media producers and outlets. \\
% Second Paragraph
% CORE MESSAGE OF THIS PARAGRAPH:
%\todo{P2.1. The second paragraph should be about what have others been doing}
For the sake of completeness, these discussions are not entirely new: So called DeepFakes (blend word of Deep learning and Fake News) have been around quite some time. First research papers like the Face2Face approach by \citet{thiesFace2FaceRealtimeFace2020} date back to the year 2016 or earlier with Facebook's 2014 DeepFace paper \cite{taigmanDeepFaceClosingGap2014}. At latest in 2017, DeepFake pornography, often in the form of revenge porn, hit the broader public \cite{coleAIAssistedFakePorn2017}. Quickly afterwards discussions arose about the implications of these technologies in regards to the spread of fake news. Despite the proliferation of payed and \gls{oss} for easy to use faceswaps, like \gls{dfl} (2019) or the InsightFace Inswapper (2023), there are only few official cases, where this technology has been used for a signular desinformation with large concequences. However, it goes without saying, that the effect in social networks, under the radar of public control, might be much bigger. But it's important to add that these cases might not even need manipulated video to spraed misinformation. Text does also suffice, as could be seen during the Covid19 pandemic and often mentioned social media channels \cite{naeemExplorationHowFake2021}.  
%\todo{P2.2. Why is the problem important? Why was this work carried out?}
The possibilities in 2023 have three key difference in comparison to the previously described, early faceswap deepfakes: 
\begin{enumerate}
  \item Excellent A.I. powered voices cloning tools with easy access
  \item Text to image generation
  \item GPT-enabled Chat Applications
\end{enumerate}
The latter is less relevant to audio-visual content, but fuels the public oppinion about A.I. tools. The first two have drastically improved the quality and possiblities of how and what kind of synthetic media can be created. In the recent months there have been several usecases of these very specific examples in increasing frequency: In 2022 a DeepFake of President Zelensky surfaced (Figure \ref{fig:zelensky-deepfake}). 
\begin{figure}[h]
  \centering
  \includegraphics[width=0.8\textwidth]{./graphics/images/Zelensky.jpg}
  \caption{left: DeepFake of President Zelensky; right: real image of Zelensky \cite{universityofvirginiaZelenskyySurrenderHoax2022}}
  \label{fig:zelensky-deepfake}
\end{figure}
The examples of fake voices and faceswaps on social media in 2023 are innumerable and can be traced back to the availability of online survices such as Resemble.ai or Elevenlabs. This also lead to several neferious usecases. To name some examples in German context: In September 2023 a primetime news host was recreated with a fake voice in order to advertise dubious financial products (figure: \ref{fig:sievers-fake}). By using Elevenlabs' checking tool, one can quickly tell, that the voice was very likely created with the software (figure: \ref{fig:sievers-11labs}). 
\begin{figure}[h]
  \centering
  \begin{subfigure}[b]{0.45\textwidth}
    \includegraphics[width=\textwidth]{./graphics/images/sievers.png}
    \caption{fake of Christian Sievers \cite{zdfDeepfakeMitZDFModerator}}
    \label{fig:sievers-fake}
  \end{subfigure}
  \hfill
  \begin{subfigure}[b]{0.45\textwidth}
    \includegraphics[width=\textwidth]{./graphics/images/sievers-11labs.png}
    \caption{Elevenlabs audio analysis \cite{elevenlabsAISpeechClassifier}}
    \label{fig:sievers-11labs}
  \end{subfigure}
  \caption{Christian Sievers case}
\end{figure}

End of November 2023 a two Videos of German chancelor Olaf Scholz were released, where his voice and lips were faked. One was part of a commercial campaign for german yellow press newspaper \cite{dwdl.deSpringerTrommeltMit}. The second is part of an art project \cite{zdfKunstinstallationDeepfakeScholzVerkuendet}. Example images for these cases are not included, as they would be just other examples for the aforementioned. It is just very clear, that the frequency of appearance is increasing, both in the case of legitimate and illegitimate content.
An environment where both categories of content coexist is very challenging in regards of trust. For legitimate newsmakers the questions arises how it can combat desinformation and at the same time use the advancements of \gls{genai} to improve production workflows. This is a dilemma that is yet to be solved. \\
The effects of these very recent technical capabilites have not yet been studied, which is why this work attempts in doing so. In a time where the very existance of \gls{genai} raises trust issues on every kind of content, specifically of those synthetically generated any findings about synthetic media reception might be helpful in better adressing all the named issues.\\
% Third Paragraph
% CORE MESSAGE OF THIS PARAGRAPH:
%\todo{P3.1. What have you done?}
This paper tried to explore the effect on potential recepients of (partially) synthetically created or A.I. enhanced media in the specific context of a german public broadcast television news show. Before conducting a study on how recipients perceive this sort of content this work also worked on specific workflows for media producers of how to produce the content.
%\todo{P3.2. What is new about your work?}
The tested videos were carefully designed by taking into account an extensive toolchain of available open source technology, making it (theoretically) possible for every media producer to recreate similar results implement (semi)automatic workflows for their media production and conduct further experiments. However the specific code implementation won't be directly shared as the risk of misuse of this project should be reduced. \\
% Fourth paragraph
% CORE MESSAGE OF THIS PARAGRAPH:
\todo{P4.1. What did you find out? What are the concrete results?}
\todo{P4.2. What are the implications? What does this mean for the bigger picture?}

\chapter{Related Work: 6 Seiten}
Trust in news is explored. 
But there is not too much about trust and credibility in news with ai footage.
Law findings. 
Verletzung Persönlichkeitsrecht.

Not too much implementation Details about the used technology

\todo{reception of media}
\todo{synthetic media reception}
\todo{uncanny valley}

\chapter{Implemented Technologies: 10 Seiten}
Docker based deployment, CI/CD. Remote work on HFF Workstation

\section{Voice Cloning Toolkit 1}

\subsection{Preparation and Pre-Processing: 1 Seite}
\subsection{Text To Speech: 2 Seiten}
\subsection{Retrieval Based Voice Conversion: 1 Seite}
\section{Lip remapping: 1 Seite}
Wav2Lip
\section{stable Diffusion + Video: 2 Seiten}
\section{Game-Engine: Unreal Engine: 2 Seiten}
Einzelpraktikum stuff

\chapter{Study: 10 Seiten}
\section{Study Design: 3 Seiten}
Spectrum of syntheticnes.
Randomization.
AB Groups. 
Sequence bias.
Basic Questions: Media usage, media trust, ai usage and trust

\section{Procedure and Participants: 1 Seite}
Online survey, spread via various channels.

\section{Results: 6 Seiten}

\chapter{Discussion: 2 Seiten}

\chapter{Conclusion: 2 Seiten}
\label{sec:conclusion}

Future implementation und Haushandlungsprozess. Was wichtig ist:) 

To close the circle with the beginning. This work might be too early. The disruptive process has just begun, the developments can happen rather quickly. It therefore might remain interesting to closely monitor how the credibility and trust toward media, both synthetic and real, unfolds in the next years. 





\todo{Outlook}

\printbibliography

All links were last followed on \today{}.

\appendix
%\input{latexhints/latexhints-english}

\pagestyle{empty}
\renewcommand*{\chapterpagestyle}{empty}
\Affirmation
\end{document}
